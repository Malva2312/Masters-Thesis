\chapter{Literature Review} \label{chap:lreview}

The main goal of this literature review is to define a strategy for completing the state of research on information fusion-based models for lung nodule characterization, with a focus on identifying fusion techniques currently applied in this field. In particular, it has three main objectives: to gain knowledge by analyzing the different types of techniques that have been used in nodule characterization, to discover the specific methods that have been designed to automate nodule characterization and to evaluate the effectiveness of these methods based on the results obtained in the respective studies. However, in addition to achieving these objectives, the review will help to understand the current state of information fusion methods in this area and will lead to a better understanding of the various approaches.

Through recent studies analyses, we will study the most widely adopted techniques, as well as hybrid approaches that combine various strategies. In addition, methods used for automatic nodule characterization will also be analyzed, with a focus on deep learning architectures adapted to CT scan image analysis. This synthesis aims to establish a comprehensive basis for future studies in the characterization of pulmonary nodules and to guide the development of more effective, interpretable and applicable models in a clinical environment.

\section{Eligibility Criteria} \label{criteria} 

In order to achieve the relevance and rigor of the selected studies, the criteria for the systematic review were specified. These criteria seek to encompass the entire body of research that has been conducted between shallow and deep feature extractors and information fusion techniques in the characterization of CT scans, mainly related to lung nodules. To expand the scope, a more extensive view of applicable methodologies was included for medical conditions that use CT technologies if the studies presented relevant approaches.

In terms of eligibility, only studies published in the last five years (2019 - 2024) were considered, which ensures that the review reflects recent advances in fusion techniques in the field of medical imaging. Articles were limited to those published in English to maintain consistency and accessibility. In addition to the characterization of pulmonary nodules, studies aimed at diagnosing other medical conditions revealed by CT scans were also included, as long as they used methodologies that could be used within the scope of this study.

\section{Search Strategy}

In this review, a comprehensive search strategy was formulated to find relevant studies in various reliable databases. The search was conducted mainly in three academic databases: IEEE Xplore, PubMed and Google Scholar.

The search process employed a set of keywords and Boolean operators to develop comprehensive queries. The primary keywords included terms such as “lung nodule characterization,” “information fusion,” “shallow feature extraction,” “deep feature extraction,” and “CT scan analysis.” Additionally, secondary terms were included to capture more specific methodologies and techniques, such as “texture features,” “shape features,” “convolutional neural networks (CNN),” and “medical image classification.”

Furthermore, in addition to the database searches, reference chaining was used to expand the list of studies. Specifically, the references of key articles excavated in the initial phase were subjected to scrutiny. As an example, significant references were extracted from the influential article Fusion of Textural and Visual Information for Medical Image Modality Retrieval Using Deep Learning-Based Feature Engineering~\cite{iqbal_fusion_2023}, which has been helpful in the study of fusion techniques for the analysis of pulmonary nodules. Reviewing the references cited in this document helped to identify other studies relevant to the objectives of the initial work, linked by common topics.



\section{Screening Process}

After the search drew out a set of studies that might be relevant, the title and abstract screening process followed. During this phase, each study's titles and abstracts were reviewed to weed out studies that are obviously not relevant. This first screening made it possible to discard the articles that didn't fit the core inclusion criteria, including those that are not associated with CT-based medical imaging or those focusing on work unrelated to the use of shallow or deep features extractors or information fusion methods.

This strategy was used to cut out irrelevant studies in a fast and easy way as well as to let potentially related papers through to the full-text analysis phase. At this stage, the objective was to screen only titles and abstracts. This enabled the easy inclusion of studies that were worth further investigation in subsequent stages.

\newpage
\section{Summary}

This literature review lays the foundations for the study and research into lung nodule characterization models based on information fusion. With the implementation of a search strategy, this review presents a current selection of studies focused on the characterization of lung nodules and including other diseases based on CT scans. The structured procedure of using the main databases and linking references ensures that relevant studies are obtained, and the selection is further boosted by the use of the title and abstract screening process, which shortens the selection to studies that are actually dealing with research in terms of deep learning, shallow feature extraction and information fusion techniques.
