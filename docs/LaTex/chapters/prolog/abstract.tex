%% abstract.tex: abstract in PT and EN  (FEUP regulations)
%% -------------------------------------------------------
\chapter*{Resumo}
%
O cancro do pulmão continua a ser uma das principais causas de mortalidade relacionada com o cancro em todo o mundo, principalmente devido aos desafios do diagnóstico tardio e às complexidades envolvidas na diferenciação entre nódulos pulmonares benignos e malignos. Esta tese introduz uma nova estrutura para a caraterização de nódulos pulmonares que integra técnicas de fusão de informação, combinando representações de aprendizagem profunda com caraterísticas radiológicas artesanais. O objetivo é melhorar o desempenho da classificação, a interpretabilidade do modelo e a fiabilidade clínica.

A metodologia proposta utiliza uma estratégia de fusão que combina caraterísticas profundas extraídas de CNNs com caraterísticas superficiais, como descritores de textura, forma e intensidade. Estas caraterísticas são integradas em vários níveis da arquitetura da rede para avaliar os efeitos de várias fases de fusão no desempenho. O sistema é rigorosamente avaliado utilizando o conjunto de dados LIDC-IDRI através de experimentação extensiva, incluindo estudos de ablação, seleção de caraterísticas e análises de explicabilidade do modelo utilizando Grad-CAM e SHAP.

Os resultados indicam que a integração de caraterísticas artesanais e profundas conduz a um melhor desempenho de classificação em diferentes arquitecturas, com melhorias consistentes nas métricas AUC. Além disso, as análises de explicabilidade destacam que as caraterísticas artesanais - especialmente os descritores de forma - desempenham um papel significativo na previsão de malignidade. Em geral, os resultados sugerem que esta abordagem baseada na fusão não só aumenta a precisão da previsão, como também melhora a transparência do modelo, proporcionando uma ferramenta mais fiável e interpretável para o diagnóstico do cancro do pulmão assistido por computador.

Esta investigação alinha-se com os esforços globais para promover a deteção precoce e garantir o acesso equitativo aos cuidados de saúde, contribuindo para o desenvolvimento de ferramentas de diagnóstico alimentadas por IA que ajudam na tomada de decisões clínicas informadas e promovem os objectivos de cuidados de saúde sustentáveis.

\chapter*{Abstract}
Lung cancer continues to be one of the leading causes of cancer-related mortality worldwide, primarily due to the challenges of late diagnosis and the complexities involved in differentiating between benign and malignant pulmonary nodules. This thesis introduces a novel framework for lung nodule characterisation that integrates information fusion techniques, combining deep learning representations with handcrafted radiomic features. The goal is to enhance classification performance, model interpretability, and clinical reliability.

The proposed methodology employs a fusion strategy that blends deep features extracted from CNNs with shallow features, such as texture, shape, and intensity descriptors. These features are integrated at multiple levels of the network architecture to evaluate the effects of various fusion stages on performance. The system is rigorously assessed using the LIDC-IDRI dataset through extensive experimentation, including ablation studies, feature selection, and model explainability analyses utilising Grad-CAM and SHAP.

The results indicate that the integration of handcrafted and deep features leads to improved classification performance across different architectures, with consistent enhancements in AUC metrics. Additionally, the explainability analyses highlight that handcrafted features - especially shape descriptors - play a significant role in predicting malignancy. Overall, the findings suggest that this fusion-based approach not only boosts predictive accuracy but also enhances model transparency, providing a more reliable and interpretable tool for computer-aided lung cancer diagnosis.

This research aligns with global efforts to promote early detection and ensure equitable access to healthcare, contributing to the development of AI-powered diagnostic tools that aid in informed clinical decision-making and advance the objectives of sustainable healthcare.