\chapter{Methodology planning }\label{chap:chap4}
In summary, this workplan outlines a systematic and rigorous approach to developing a robust pulmonary nodule classification system. By integrating advanced deep learning models with complementary shallow feature extractors and employing sophisticated fusion methods, the study aims to achieve significant improvements in diagnostic accuracy and robustness, with clear pathways for addressing potential limitations and risks.
%% Detailed Work Plan / Activity cards
    %% Task Identification
    %% Duration, begin and end dates
    %% Goals and expected outcomes
    %% Detailed description (may include materials and methods)
    %% Deliverables
    %% Milestones
%% Description
%% Experimental process & Evaluation Design
    %% Introduction
    %% Research design
    %% Research context
    %% Population and sampling
    % LIDC-IDRI as example
%% Expected Results

%% Conclusion
    %% Ethical considerations
    
    %% Problem and goals
    %% Limitations Acknowledge
    
    %% Conclusions drawn from the related work and gap analysis
    
    %% SMART analysis of the project goals
    %% SWOT analysis of the project proposal
    
    
    %% Mitigation plan
    %% Risk assessment and contingency plan
\section{Population and sampling}
    The lung nodule classification study will use a population of \ac{ct} images from publicly available and clinically validated datasets, generally used for this purpose. We will fully trust in their assignment of labels (benign/malignant) to conduct experiments. 
    
    To retrieve insights into the characteristics and quality of the data we will submit it to an analysis and provide some visualization for easier interpretation. Computing and summarizing statistics such as mean, median, and standard deviation, for nodule size, patient demographics, and other clinical variables and distribution of classifications in the used datasets to evaluate balance and ascertain the need to perform sampling techniques. For each dataset selected, pre-processing will be carried out in order to improve the quality of the images: ensuring uniformity if necessary, such as applying resampling techniques, and removing, for example, noise and other irrelevant information.
    
    In order to proceed with the experimentation phase, the data will be randomly divided into training and test sets, with the aim of carrying out evaluations afterward. This division will mitigate possible over-fitting effects.
        
\section{Evaluation Strategy}
Each combination will be evaluated using standardized cross-validation protocols to ensure robust and generalizable results. Detailed logs of the performance metrics for each run will be maintained to enable comprehensive comparative analysis.
Accuracy provides an overall measure of the model’s correctness by quantifying the proportion of correctly classified cases. However, accuracy alone can be misleading in imbalanced datasets, where the majority class dominates predictions. Sensitivity (recall) is essential in this context, as it measures the model's ability to correctly identify malignant nodules, ensuring minimal false negatives, which is critical for early cancer detection and patient prognosis. Conversely, specificity assesses the model’s capability to correctly classify benign nodules, reducing false positives and preventing unnecessary invasive procedures. Finally, the area under the receiver operating characteristic curve (AUC-ROC) provides a comprehensive assessment of the model's discriminative ability across various decision thresholds, balancing sensitivity and specificity.

\section{Deep-Learning Selection}
The first step in the research is to detect and manifest deep learning models that are the best of the best and specifically designed for the classification of pulmonary nodules in \ac{ct} images. The primary target is to create a baseline for performance measurements that are to be taken into consideration. The retrieve models will be subject of a defined evaluation and ranked based on performance. We will use the best performing models to combine them with additional methods and features. 

\section{Shallow Extractors}
In parallel with deep learning, shallow feature extractors are employed to capture complementary visual and textural information about pulmonary nodules. This stage aims to identify and implement diverse extractors that contribute unique feature sets. We will select the most used extractors for texture, shape, moment-based, gradient-based, and hierarchical features used in the state of the art.
The objective is to use them in the same dataset used for deep learning methods and ensuring compatibility with fusion techniques.

\section{Fusion Methods}
The integration of deep learning outputs with shallow feature sets is crucial for leveraging the strengths of both approaches. To achieve this, state-of-the-art fusion techniques will be examined and implemented. Fusion can be applied at two primary levels: decision-level and feature-level. Decision-level fusion combines predictions from multiple models or classifiers using techniques such as weighted voting, majority voting (MAX-VOTE), and prediction averaging, allowing the ensemble of predictions to refine the overall accuracy. On the other hand, feature-level fusion involves concatenating or averaging feature vectors from deep learning models and shallow extractors, often with dimensionality reduction techniques like \ac{PCA} to manage high-dimensional data. These techniques will be implemented in a modular framework, ensuring compatibility with various model configurations.

\section{Experimentation}
The experimentation phase is designed to systematically evaluate the performance of different combinations of deep learning models, shallow feature extractors, and fusion techniques in classifying pulmonary nodules. A flexible workflow will be developed to facilitate the testing of these combinations. This workflow begins by selecting one of the three top-performing deep learning models identified earlier, followed by choosing one to all five shallow feature extractors. The selected outputs are then combined using one of the chosen fusion methods. The pipeline processes the input \ac{ct} images, extracts features, applies fusion, and trains the combined model. For each configuration, the model's performance will be assessed using metrics such as accuracy, sensitivity, specificity, F1-score, area under the curve (AUC), and false-positive rate (FPR).
This systematic approach ensures that all potential interactions between deep learning models, shallow feature extractors, and fusion techniques are explored. The goal is to identify the configurations that deliver the highest classification accuracy, outperforming both standalone deep learning models and traditional classifiers. The experimental results will provide valuable insights into the synergies between deep and shallow methods, establishing a robust foundation for future advancements in pulmonary nodule classification.

\section{Expected Results}

We anticipate that the proposed model will demonstrate significant improvements over baseline methods. Specifically, the study expects: to improve the accuracy, sensitivity, specificity, and AUC-ROC of defined baselines and other state of the art approaches; to achieve a robust performance on unseen datasets, indicating effective generalization; to prove the combined use of deep learning and shallow features will result in a model with an higher discriminative ability; to enhance interpretability and manage computational demands through design and potential use of specialist models.


\section{Conclusion}

The research will exclusively use publicly available datasets, ensuring compliance with ethical standards and privacy guidelines. Since no new patient data will be collected and because no sensitive data is located in the datasets, the ethical risks are negligible.

We knowledge that the research could be held back due to parameter configurations difficulties, challenges on accurately segmenting complex nodule cases and the possibility that the selected data sets might not fully represent the diversity found in the clinical contexts required for implementation in diagnostics.


To address potential challenges, techniques will be applied to mitigate overfitting and in case of training or data challenges, the approach includes the flexibility to simplify the model or adopt alternative (or simpler) methods.
Other risks, such as delays, dataset issues, or training difficulties will be monitored, in order to mitigate them and resolve problems before they cause major delays and other conflicts that are more difficult to predict and control.

\subsection{SWOT Analysis}
\begin{tcbraster}[raster columns=2, boxrule=0mm, arc=0mm]
\begin{tcolorbox}[equal height group=A, size=fbox, colback=swotS!60, colframe=swotS!80!black, title=\textsc{strengths}]
\begin{enumerate}
    \item Use of deep and shallow extractors fused with advanced strategies.
    \item Access to clinically validated, publicly available datasets.
    \item Rigorous evaluation protocols.
\end{enumerate}
\end{tcolorbox}
\begin{tcolorbox}[equal height group=A, size=fbox, colback=swotW!60, colframe=swotW!80!black, title=\textsc{weaknesses}]
\begin{enumerate}
    \item Highly dependent on training set demographics representation.
    \item High computational resource requirements.
    \item Complexity in integrating multiple methodologies.
\end{enumerate}
\end{tcolorbox}
\begin{tcolorbox}[equal height group=B, size=fbox, colback=swotO!60, colframe=swotO!80!black, title=\textsc{opportunities}]
\begin{enumerate}
    \item Development of novel fusion strategies and architectures.
    \item Contribution to advancements in CAD systems.
\end{enumerate}
\end{tcolorbox}
\begin{tcolorbox}[equal height group=B, size=fbox, colback=swotT!60, colframe=swotT!80!black, title=\textsc{threats}]
\begin{enumerate}
    \item New competing methods and research in the field.
    \item Variability in image data that could impact model performance.
    \item Compliance with clinical and regulatory requirements.
\end{enumerate}
\end{tcolorbox}
\end{tcbraster}

\subsection{Work Plan}
\begin{ganttchart}[
    hgrid,
    vgrid,
    title height=1,
    bar height=0.6
]{0}{20}
    \gantttitle{Work Plan by Week}{21} \\
    \gantttitlelist{0,...,20}{1}{} \\

    % SOTA and Data Collection Phase
    \ganttbar{SOTA Revision and Improvement}{0}{1} \\
    \ganttmilestone{SOTA Finalized}{1} \\
    \ganttbar{Data Selection and Sampling}{1}{2} \\
    \ganttbar{Data Analysis and Visualization}{2}{3} \\
    
    % Data Pre-Processing Phase
    \ganttbar{Data Pre-Processing}{3}{4} \\
    \ganttmilestone{Prepared Data}{4} \\

    % Evaluation and Model Development
    \ganttbar{Evaluation Workflow Development}{4}{6} \\
    \ganttbar{Deep Learning Selection}{5}{6} \\
    \ganttbar{Shallow Feature Extraction}{6}{7} \\
    \ganttbar{Fusion Methods Selection}{7}{8} \\
    \ganttmilestone{Experiment Subjects Finalized}{8} \\

    % Experimentation and Tuning
    \ganttbar{Experimentation Phase}{8}{16} \\
    \ganttbar{Model Tuning}{10}{16} \\
    \ganttmilestone{Experimentation Completed}{16} \\

    % Thesis Writing and Final Adjustments
    \ganttbar{Thesis Writing Draft}{7}{16} \\
    \ganttmilestone{Thesis Draft Finalized}{16} \\
    \ganttbar{Contingencies and Final Adjustments}{16}{20} \\
    \ganttmilestone{Final Adjustments Completed}{20} \\
    
\end{ganttchart}



